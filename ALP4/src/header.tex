\documentclass[
	10pt,								% globale Schriftgröße
	parskip=half-,						% setzt Absatzabstand hoch
	paper=a4,							% Format
	english,ngerman,					% lädt Sprachpakete
	]{scrartcl}							% Dokumentenklasse

% //////////////////// Pakete laden ////////////////////
\usepackage{amsmath}			% MUSS vor fontspec geladen werden
\usepackage{mathtools}			% modifiziert amsmath
\usepackage{amssymb}			% mathematische symbole, für \ceckmarks
\usepackage{amsthm}				% für proof
\usepackage{mathrsfs}			% für \mathscr
\usepackage{latexsym}
\usepackage{marvosym}				% für Lightning

\usepackage{fontspec} 			% funktioniert nur mit den neueren Compilern z.B. XeLaTeX
\usepackage{microtype}			% für bessere Worttrennung
\usepackage[ngerman]{babel} 	% Spracheinstellung
\usepackage{lmodern}			% verändert verwendete Schriftart, damit sie weniger pixelig ist

\usepackage{verbatim}
\usepackage{listings}			% Für Quellcode

\usepackage{graphicx}
\usepackage{tabularx}			% für Tabellen mit gleicher Spaltenbreite und automatischen Umbrüchen
\usepackage{fullpage}
\usepackage{multirow}			% für multirow in tabulars
\usepackage{rotate}
\usepackage[cmyk,table]{xcolor} % um Farben zu benutzen, kann mehr als das Paket color
\usepackage[					% Verlinkungen
	colorlinks,					% farbige Schrift, statt farbiger Rahmen
	linktocpage,				% verlinkt im Abb.Verzeichnis Seitenzahl statt Bildunterschrift
	linkcolor=blue				% setzt Farbe der Links auf blau
	]{hyperref}					% nur für digitale Anwendungen, url = "http://www.example.com"
\usepackage{url}				% für Webadressen wie e-mail usw.: "\url{http://www.example.com}"

\usepackage{enumerate}			% für versch. Aufzählungezeichen wie z.B. a)
\usepackage{xspace}				% folgt ein Leerzeichen nach einem \Befehl, wird es nicht verschluckt.
\usepackage{cancel}				% für das Durchstreichen u.a. in Matheformeln mit \cancel
\usepackage{float}              % zum Forcieren der Position von figure-Umgebungen

% zum Zeichnen (u.a. von Graphen)
\usepackage{fp}
\usepackage{tikz}
\usetikzlibrary{tikzmark}			% für \tikzmark{toRemember}
\usetikzlibrary{positioning}	% verbesserte Positionierung der Knoten
\usetikzlibrary{automata}		% für Automaten (GTI)
\usetikzlibrary{arrows}
\usetikzlibrary{shapes}
\usetikzlibrary{decorations.pathmorphing}
\usetikzlibrary{decorations.pathreplacing}
\usetikzlibrary{decorations.shapes}
\usetikzlibrary{decorations.text}

% //////////////////// Syntaxhighlighting ////////////////////
\lstloadlanguages{Python, Haskell, [LaTeX]TeX, Java}
\lstset{
   basicstyle=\footnotesize\ttfamily,	% \scriptsize the size of the fonts that are used for the code
   backgroundcolor = \color{bgcolour},	% legt Farbe der Box fest
   breakatwhitespace=false,	% sets if automatic breaks should only happen at whitespace
   breaklines=true,			% sets automatic line breaking
   captionpos=t,				% sets the caption-position to bottom, t for top
   commentstyle=\color{codeblue}\ttfamily,% comment style
   frame=single,				% adds a frame around the code
   keepspaces=true,			% keeps spaces in text, useful for keeping indentation
							% of code (possibly needs columns=flexible)
   keywordstyle=\bfseries\ttfamily\color{codepurple},% keyword style
   numbers=left,				% where to put the line-numbers;
   							% possible values are (none, left, right)
   numberstyle=\tiny\color{codegreen},	% the style that is used for the line-numbers
   numbersep=5pt,			% how far the line-numbers are from the code
   stepnumber=1,				% nummeriert nur jede i-te Zeile
   showspaces=false,			% show spaces everywhere adding particular underscores;
							% it overrides 'showstringspaces'
   showstringspaces=false,	% underline spaces within strings only
   showtabs=false,			% show tabs within strings adding particular underscores
   flexiblecolumns=false,
   tabsize=1,				% the step between two line-numbers. If 1: each line will be numbered
   stringstyle=\color{orange}\ttfamily,	% string literal style
   numberblanklines=false,				% leere Zeilen werden nicht mitnummeriert
   xleftmargin=1.2em,					% Abstand zum linken Layoutrand
   xrightmargin=0.4em,					% Abstand zum rechten Layoutrand
   aboveskip=2ex, 
}

\lstdefinestyle{py}{
   language=Python,
}
\lstdefinestyle{hs}{
   language=Haskell,
}
\lstdefinestyle{tex}{
	language=[LaTeX]TeX,
	escapeinside={\%*}{*)},     % if you want to add LaTeX within your code
	texcsstyle=*\bfseries\color{blue},% hervorhebung der tex-Schlüsselwörter
	morekeywords={*,$,\{,\},\[,\],lstinputlisting,includegraphics,
	rowcolor,columncolor,listoffigures,lstlistoflistings,
	subsection,subsubsection,textcolor,tableofcontents,colorbox,
	fcolorbox,definecolor,cellcolor,url,linktocpage,subtitle,
	subject,maketitle,usetikzlibrary,node,path,addbibresource,
	printbibliography},% if you want to add more keywords to the set
     numbers=none,
     numbersep=0pt,
     xleftmargin=0.4em,
}

\lstdefinestyle{java}{
	language=Java,
	extendedchars=true,		% lets you use non-ASCII characters;
   						% for 8-bits encodings only, does not work with UTF-8
}

\lstdefinelanguage[x64]{Assembler}     % add a "x64" dialect of Assembler
   [x86masm]{Assembler} % based on the "x86masm" dialect
   % with these extra keywords:
   {morekeywords={CDQE,CQO,CMPSQ,CMPXCHG16B,JRCXZ,LODSQ,MOVSXD, %
                  POPFQ,PUSHFQ,SCASQ,STOSQ,IRETQ,RDTSCP,SWAPGS, %
                  rax,rdx,rcx,rbx,rsi,rdi,rsp,rbp, %
                  r8,r8d,r8w,r8b,r9,r9d,r9w,r9b}
}					% for 8-bits encodings only, does not work with UTF-8

\lstdefinestyle{c}{
	language=c,
	extendedchars=true,		% for 8-bits encodings only, does not work with UTF-8
}

% //////////////////// eigene Kommandos ////////////////////
\newcommand\FU{Freie Universität Berlin\xspace}% benötigt package xspace
\newcommand\gdw{g.\,d.\,w.\xspace}
\newcommand\oBdA{o.\,B.\,d.\,A.\xspace}
\newcommand{\Eu}{\texteuro}
\newcommand\N{\mathbb{N}\xspace}
\newcommand\Q{\mathbb{Q}\xspace}
\newcommand\R{\mathbb{R}\xspace}
\newcommand\Z{\mathbb{Z}\xspace}
\newcommand\ohneNull{\ensuremath{\backslash\lbrace 0\rbrace}}% \{0}
\let\dhALT\dh	% Schreibt Befehl \dh in \dhALT um
\renewcommand\dh{d.\,h.\xspace}	%renew überschreibt command \dh
\newcommand\Bolt{\;\text{\LARGE\raisebox{-0.3em}{\Lightning}\normalsize}\xspace}% Blitz
\newcommand\zz{\ensuremath{\raisebox{+0.25ex}{Z}% zu zeigen
			\kern-0.4em\raisebox{-0.25ex}{Z}%
			\;\xspace}}
\newcommand{\from}{\ensuremath{\colon}}
\newcommand{\floor}[1]{\lfloor{#1}\rfloor}
\newcommand{\ceil}[1]{\lceil{#1}\rceil}
 \renewcommand{\L}{\ensuremath{\mathcal{L}}\xspace}
 \renewcommand{\P}{\ensuremath{\mathcal{P}}\xspace}
 \newcommand{\NL}{\ensuremath{\mathcal{N}\kern-0.2em\mathcal{L}}\xspace}
 \newcommand{\NP}{\ensuremath{\mathcal{NP}}\xspace}

% //////////////////// Mathefunktionen ////////////////////
\DeclareMathOperator{\Landau}{\mathcal{O}}
\DeclareMathOperator{\True}{True}
\DeclareMathOperator{\False}{False}

% //////////////////// eigene Theoreme ////////////////////
\newtheorem{theorem}{Satz}
\newtheorem{corollary}[theorem]{Folgerung}
\newtheorem{lemma}[theorem]{Lemma}
\newtheorem{observation}[theorem]{Beobachtung}
\newtheorem{definition}[theorem]{Definition}
\newtheorem{Literatur}[theorem]{Literatur}
% konfiguriert proof
\makeatletter
\newenvironment{Proof}[1][\proofname]{\par
  \pushQED{\qed}%
  \normalfont \topsep6\p@\@plus6\p@\relax
  \trivlist
  \item[\hskip\labelsep
%         \itshape
        \bfseries
    #1\@addpunct{.}]\ignorespaces
}{%
  \popQED\endtrivlist\@endpefalse
}
\makeatother

% //////////////////// eigene Farben ////////////////////
\let\definecolor=\xdefinecolor
\definecolor{FUgreen}{RGB}{153,204,0}
\definecolor{FUblue}{RGB}{0,51,102}

\definecolor{middlegray}{rgb}{0.5,0.5,0.5}
\definecolor{lightgray}{rgb}{0.8,0.8,0.8}
\definecolor{orange}{rgb}{0.8,0.3,0.3}
\definecolor{azur}{rgb}{0,0.7,1}
\definecolor{yac}{rgb}{0.6,0.6,0.1}
\definecolor{Pink}{rgb}{1,0,0.6}

\definecolor{bgcolour}{rgb}{0.97,0.97,0.97}
\definecolor{codegreen}{rgb}{0,0.6,0}
\definecolor{codegray}{rgb}{0.35,0.35,0.35}
\definecolor{codepurple}{rgb}{0.58,0,0.82}
\definecolor{codeblue}{rgb}{0.4,0.5,1}

% //////////////////// eigene Settings ////////////////////

\textheight = 230mm		% Höhe des Satzspiegels / Layouts
\footskip = 10ex			% Abstand zw. Fußzeile und Grundlinie letzter Textzeile
\parindent 0pt			% verhindert Einrückung der 1. Zeile eines Absatzes
\setkomafont{sectioning}{\rmfamily\bfseries}% setzt Ü-Schriften in Serifen, {disposition}